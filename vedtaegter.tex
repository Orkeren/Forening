\documentclass[a4paper,11pt,danish]{article}
\usepackage[T1]{fontenc}
\usepackage[utf8]{inputenc}
\usepackage{babel}
\usepackage[pdftex]{graphicx}

\title{Vedtægter for DIKUrevyen}
\author{Jenny-Margrethe Vej, Phillip Alexander Roschnowski, \\Ronni Elken Lindsgaard \& Søren Lynnerup}
\date{D. 03/01/2012}

\addtolength{\textheight}{100pt}
\topmargin=-50pt

\def\thesection{\S\arabic{section}}
\def\thesubsection{\textbf{Stk. \arabic{subsection}} }

\def\stk{
    \vspace{0.5em}
    \addtocounter{subsection}{1}
    \noindent\thesubsection \relax
}

\begin{document}

\maketitle

\begin{center}
    \includegraphics[width=5cm]{logo.png}
\end{center}\\


\newpage

\section{Foreningens navn}
Foreningens navn er "DIKUrevyen".


\section{Foreningens hjemsted}
Foreningen holder til på Københavns Universitet i de lokaler, der er stillet Datalogisk Institut til rådighed. 

\section{Foreningens formål}
Foreningen har til formål at at lave studenterrevy en gang om året, herunder afholdelse af relaterede og forberedende arrangementer såsom forfattermøder, hytteture og legedage. \\

\noindent Foreningen har ligeledes til formål at fremme samarbejdet med andre grupperinger, der arbejder hen imod samme mål, heriblandt 'Harlem Klub', som er introgruppen på Datalogisk Institut.  

\section{Medlemmer}
Enhver, som er tilmeldt foreningens informationsmail revy@diku.dk, og som tilslutter sig foreningens formål, er medlem af foreningen.\\

\noindent Ekskludering eller udmelding af foreningen foregår ved afmeldelse af mailinglisten revy@diku.dk. Mailinglisten administreres til enhver tid af ledelsen. \\

\noindent Mailinglistens medlemmer dokumenteres med en udskrift af listen en gang om måneden. I tilfælde af nedbrud eller tab af data på anden vis, er det den sidst printede liste, der er aktuel.

\section{Generalforsamling}
Generalforsamlingen afholdes en gang årligt i september måned, og indvarsles ved almindelig annoncering på foreningens informationsmail senest 14 dage før. \\

\noindent Dagsordenen for Generalforsamlingen skal mindst indeholde følgende punkter: 

\begin{enumerate}
\item Ledelsens beretning om foreningens gang det forløbne år
\item Ledelsen fremlægger det nye års arbejdsmetoder, som er baseret på årets evalueringer
\item Regnskabet fremlægges og godkendes
\item Budget for næste år fremlægges og godkendes
\item Eventuelt
\end{enumerate}

\noindent Forslag til generalforsamlingen sendes til revyboss@diku.dk senest 7 dage før generalforsamlingen til ledelsens godkendelse. Ledelsen fremsætter herefter en endelig dagsorden, som offentliggøres senest 2 dage før generalforsamlingen. 

\section{Ekstraordinær generalforsamling}
Har ledelsen behov for at annoncere ekstraordinære meddelelser indkaldes der til ekstraordinær generalforsamling med mindst 14 dages varsel. \\

\noindent Ønsker medlemmer at indkalde til Ekstraordinær generalforsamling, anmodes om dette via mail til revyboss@diku.dk.

\section{Afstemning}
Til generalforsamlingen træffes beslutninger ved simpelt flertal, hvilket vil sige mere end halvdelen af de afgivne stemmer. Stemmeberettigede er personer, der inden for det sidste år, har været medlem af foreningen, og som er til stede ved generalforsamlingen. Dette dokumenteres ved hjælp af de udtræk, der foretages hver måned, jf. punkt 4. Beslutninger kan foretages ved håndsoprækning. Hvis nogen ønsker det, skal der foretages skriftlig afstemning.

\subsection{Skriftlig afstemning}
Blanke stemmer anses som ugyldige og afgivne, det vil sige, de tæller med i antallet af afgivne stemmer. 
En stemmeseddel er ugyldig: 

\begin{itemize}
\item Hvis den indeholder mere end et ja eller nej
\item Hvis der benyttes andre sedler end de udleverede stemmesedler
\item Hvis der benyttes tegn eller påskrives bemærkninger, som kan afsløre stemmeafgiverens identitet
\end{itemize}

\noindent En ugyldig stemme anses som en afgiven stemme.

\section{Ledelsen}
Ledelsen består af revybosserne. Revybosserne består af 2-3 medlemmer, som har en ekstra interesse i, at føre revyen videre. Ledelsen er selvsupplerende.\\

\noindent Ledelsen har til ansvar, at tage beslutninger i foreningens bedste, og ikke i personlig interesse, når der opstår opgaver, stillingsspørgsmål og andet vedrørende foreningen. Ligeledes har de pligt til at lytte til foreningens øvrige medlemmer. \\

\section{Foreningens økonomi og ejendom}
\subsection{Foreningens ejendomme}
Foreningens ejendomme inkluderer revyens formue, teknisk udstyr mm.\footnote{Se HTML dokument\_1}.

\subsection{Økonomiansvarlig}
Varetagelsen af foreningens konto foretages af ledelsen, nærmere betegnet den økonomiansvarlige i ledelsen. Den økonomiansvarlige har til pligt at sørge for, foreninges formue forvaltes på bedste vis. \\

\noindent Den økonomiansvarlige har fuldmagt til foreningens konto, og kan ligeledes administrere foreningens Dankort. 

\subsection{Hæftelse}
Foreningen hæfter kun for sine forpligtelser med den af foreningen til enhver tid tilhørende formue, herunder foreninges ejendomme. Der påhviler ikke foreningens medlemmer eller ledelsen nogen personlig hæftelse.

\subsection{Regnskabsår}
Foreningens regnskabsår løber fra 1. september til 30. august. 

\section{Ændringer af vedtægter}
Ændringer til vedtægterne kan kun ske på ordinær generalforsamling med mindst 2/3 af de stemmeberettigedes accept, dog med mindst 15 stemmer.

\section{Materiale}
Udsendes der materiale på foreningens mailinglister eller lægges materiale i det interne arkiv, overgives samtidig brugsretten til foreningen med henblik på offentliggørelse i foreningens offentlige dokument og videoarkiv efter udvælgelse til forestilling.

\section{Opløsning}
Beslutning om foreningens opløsning kan træffes på en ekstraordinær generalforsamling. Formålet med generalforsamlingen (i.e. opløsning) skal fremgå af indkaldelsen. Foreningen kan besluttes opløst med mindst 2/3 af de stemmeberettigedes accept, dog med mindst 15 stemmer. Er fremmødet på under 15 personer, kan foreningen besluttes opløst ved enstemmigt flertal blandt de stemmeberettigede.\\

\noindent Nedlæggelse af foreningen, herunder fordeling af formue og anden ejendom, kan tidligst træde i kraft 14 dage efter den opløsende generalforsamling.\\

\noindent Ønsker en gruppe på mindst 10 personer at fortsætte foreningens virke, skal de anmode herom inden de 14 dage er gået. Medmindre der kan rejses betydelig tvivl eller mistillid ift. gruppens evner til at arbejde hen imod foreningens formål, annulleres opløsningen øjeblikkeligt, og gruppen får til ansvar at drive foreningen videre.\\

\noindent Udvises der interesse for overtagelse af foreningens virke fra mere end en enkelt gruppe, har den afgående ledelse til ansvar at afgøre konflikten.\\

\noindent Foreningens midler skal ved opløsning uddeles til andre studentersociale foreninger på Datalogisk Institut eller til de andre revyer på Naturvidenskabeligt Fakultet. Dette gælder også, selvom fakultetet skifter navn, eller fusioneres med andre fakulteter. \\

\noindent Har foreningen midler, der er ejet i fælleskab med andre, overdrages foreningens ejerskab til de resterende medejere. 

\section{Ikrafttrædelse}
Vedtægterne træder i kraft 1. marts 2012 ved foreningens stiftelse. \\

\noindent Vedtægterne er sidst revideret 28. februar 2012.

\end{document} 