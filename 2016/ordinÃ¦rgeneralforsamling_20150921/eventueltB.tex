\documentclass[a4paper,11pt]{report}

\usepackage[utf8]{inputenc}
\usepackage[T1]{fontenc}
\usepackage[danish]{babel}
\usepackage{url}
\usepackage[toc,page]{appendix}

\newenvironment{quotationb}%
{\begin{tabular}{|p{10cm}}}%
{\\\end{tabular}}

\begin{document}

\section*{Eventuelt B til ordinær generalforsamling 21. september 2015.}

\subsection*{Vedtægtsændringsforslagstilføjelse: Undgå misbrug af ekskludering
før generalforsamling}

Som det er nu kan bestyrelsen ekskludere medlemmer uden begrundelse og med det
samme jf. § 4.  Kombineret med vedtægtsændringsforslag 6d, så bliver det
muligt at ekskludere medlemmer lige før en generalforsamling, således at de ikke
kan stemme.  Den nuværende tekst til § 7 tillader nemlig personer som har
været medlem indenfor det sidste år at stemme.

Derfor forslås en ekstra vedtægtsændring for at undgå dette scenarie.
Bestyrelsen/bosserne vil stadig kunne ekskludere uden begrundelse og med det
samme.  Og man skal stadig være medlem på det tidspunkt generalforsamlingen
finder sted.

Men med de forslåede ændringer bliver følgende muligt:

\begin{itemize}
\item At indkalde til en ekstraordinær generalforsamling, hvor følgende kan ske:
\begin{itemize}
\item Rejse et mistillidsvotum mod bestyrelsen.  Hvis det lykkes skal
bestyrelsen gennemgå § 8 stk. 1 og stk. 2, såfremt vedtægtsændringsforslag 6c
bliver vedtaget.
\item Genindmelde medlemmer der er blevet ekskluderet inde for de sidste 2
måneder.
\item Rulle vedtægtsændringer tilbage der er blevet vedtaget på en
generalforsamling, som fandt sted inden for de sidste 2 måneder.
\end{itemize}
\item Det kræver 15 medlemmers underskrift at indkalde til en ekstraordinær
generalforsamling, og underskrifter fra medlemmer, der er blevet ekskluderet
inden for de sidste 14 dage da underskriften bliver modtaget, tæller også.
\item Bestyrelsen \emph{skal} indkalde til en ekstraordinær
generalforsamling inden for 21 dage efter de har modtaget alle 15 stemmer.
\end{itemize}

Disse regler skulle forbygge bestyrelsen i at misbruge sin magt.  Forhåbentligt
bliver de aldrig nødvendige at tage i brug.

Se bilag for de konkrete vedtægtsændringer.

\begin{appendices}

\chapter{Undgå misbrug af ekskludering før generalforsamling}

\section*{Teksttilføjelse til § 4}

\begin{quotationb}
Stk. 1 Genindmelding af ekskluderede medlemmer på en generalforsamling ved
simpelt flertal\\
\\
På en generalforsamling kan medlemmer der er blevet ekskluderet inde for de
sidste 60 dage genindmeldelse ved et simpelt flertal af de stemmeberettigede
personers stemmer, jf. § 7.
\end{quotationb}

\section*{Tekstændringer til § 6}

\paragraph{Afsnit 2}

\begin{quotationb}
Medlemmerne kan kræve en ekstraordinær generalforsamling ved at sende
minimum 15 underskrifter til bestyrelsen via mail til \url{boss@dikurevy.dk}.
Underskrifter fra medlemmer, der er blevet ekskluderet inde for de sidste 14
dage siden underskriftens modtagelse, tæller også som underskrifter.
\end{quotationb}

\setlength{\parskip}{15pt}
(Fuld omskrivning af afsnit 2.)
\setlength{\parskip}{0pt}

\section*{Teksttilføjelse til § 6}

\begin{quotationb}
Modtager bestyrelsen nok underskrifter inde for en periode af 14 dage, skal
bestyrelsen indkalde til en ekstraordinær generalforsamling, der finder sted
inde for 21 dage.\\
\\
På en ekstraordinær generalforsamling kan ekskluderede medlemmer, der var
gyldige til at indkalde til den ekstraordinære generalforsamling, stemme,
jf. § 7 stk. 3.\\
\\
På en ekstraordinær generalforsamling kan der bringes et mistillidsvotum mod
bestyrelsen, jf. § 8 stk. 4, genindmeldelse af ekskluderede medlemmer, jf. § 4
stk. 1 og tilbagerulning af umiddelbart godkendte vedtægtsændringer,
jf. § 10 stk. 2.
\end{quotationb}

\section*{Teksttilføjelse til § 7}

\begin{quotationb}
Stk. 3 Umiddelbart ekskluderede medlemmers stemmerettighed på en ekstraordinær
generalforsamling\\
\\
Medlemmer som er ekskluderet, hvis underskrift blev modtaget inde for 14
dage af deres eksklusion, har fuld stemmeberettigelse ved den ekstraordinære
generalforsamling de var med til at indkalde.\\
\\
Det kræver at de ekskluderede medlemmer har deltaget i underskriftsindsamlingen
til den ekstraordinære generalforsamling.  De har også ret til at stemme in
absentia via fuldmagt, jf. stk. 2.
\end{quotationb}

\section*{Teksttilføjelse til § 8}

\begin{quotationb}
Stk. 4 Mistillidsvotum mod bestyrelsen\\
\\
Kun på en ekstraordinær generalforsamling kan medlemmerne bringe et
mistillidsvotum mod bestyrelsen.  Hvis 2/3 og minimum 15 af de stemmeberettigede
personer, jf. § 7, stemmer imod bestyrelsen, skal bestyrelsen nominere
en ny bestyrelse, jf. § 8 stk. 1, på samme generalforsamling.
\end{quotationb}

\section*{Teksttilføjelse til § 10}

\begin{quotationb}
Stk. 2 Rulle nye vedtægtsændringer tilbage\\
\\
Kun på en ekstraordinær generalforsamling kan medlemmerne rulle
vedtægtsændringer som er godkendt inde for de sidste 60 dage tilbage.  Dette kan
ske ved et simpelt flertal.
\end{quotationb}

\end{appendices}

\end{document}
