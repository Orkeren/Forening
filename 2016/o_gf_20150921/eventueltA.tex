\documentclass[a4paper,11pt]{report}

\usepackage[utf8]{inputenc}
\usepackage[T1]{fontenc}
\usepackage[danish]{babel}
\usepackage{url}
\usepackage[toc,page]{appendix}

\newenvironment{quotationb}%
{\begin{tabular}{|p{10cm}}}%
{\\\end{tabular}}

\begin{document}

\section*{Eventuelt A til ordinær generalforsamling 21. september 2015.}

\subsection*{Vedtægtsændringsforslagstilføjelse: Mulighed for
overgangsbestyrelse}

Hvis vedtægtsændringsforslag 6c bliver vedtaget, så er bestyrelsen ikke længere
selvsupplerende.  Hvilket betyder at hvis en bestyrelse ikke længere kan
udføre sine opgaver som bestyrelse, skal de første træde tilbage på en
generalforsamling.

Dette forslag tillader bestyrelsen at oprette en overgangsbestyrelse.  Denne
overgangsbestyrelse kan maksimalt sidde i 28 dage, og skal indkalde til en
ekstraordinær generalforsamling inde for 7 dage af deres udnævnelse.

På denne ekstraordinære generalforsamling skal overgangsbestyrelsen nominere
en ny bestyrelse.

Se bilag for konkrete vedtægtsændringer.

\begin{appendices}

\section*{Teksttilføjelse til § 8}

\begin{quotationb}
Stk. 3 Overgangsbestyrelse\\
\\
Kan bestyrelsen ikke udføre sin opgave som bestyrelse, kan bestyrelsen
udnævne en overgangsbestyrelse til at indkalde til en ekstraordinær
generalforsamling om en ny bestyrelse.\\
\\
En overgangsbestyrelse kan maksimalt sidde i 28 dage.  Inde for 7 dage efter
dens udnævnelse skal der være indkaldt til en ekstraordinær generalforsamling om
en ny bestyrelse.  Overgangsbestyrelsen nominerer en ny bestyrelse på
generalforsamlingen, jf. stk. 1.
\end{quotationb}

\end{appendices}

\end{document}
