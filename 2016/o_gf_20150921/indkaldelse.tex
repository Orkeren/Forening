\documentclass[a4paper,11pt]{report}

\usepackage[utf8]{inputenc}
\usepackage[T1]{fontenc}
\usepackage[danish]{babel}
\usepackage{url}
\usepackage[toc,page]{appendix}

\newenvironment{quotationb}%
{\begin{tabular}{|p{10cm}}}%
{\\\end{tabular}}

\begin{document}

\section*{Indkaldelses til ordinær generalforsamling 21. september 2015}

I henhold til foreningens vedtægter\footnote{\url{http://dikurevy.dk/vedtaegter.pdf}} indkaldes der hermed til
ordinær generalforsamling i DIKUrevyen.

Generalforsamlingen afholdes kl. 17:00 den 21. september 2015 i lokale
1-0-37 (eller et andet øvelseslokale hvis der er optaget).

Tilmelding:\\
\url{http://xn--mder-gra.dikurevy.dk/?meeting=2015-09-21}

Der er fem vedtægtsændringsforslag fra ledelsen i denne dagsorden. 
Vedtægtsændringer kræver at minimum 15 personer stemmer for.  Så det er
\emph{meget} vigtigt at I deltager, hvis I er interesseret i
vedtægtsændringsforslagene (se dagsordenens punkt~\ref{itm:changes}).

\section*{Dagsorden}

\begin{enumerate}
\item Valg af dirigent.
\item Fremlægges af foreningens gang det forløbne år.
\item Opfølgning fra krisemødet og postrevyen.
\item Regnskab.
\item Budget for revyåret 2016.
\item \label{itm:changes} Vedtægtsændringer.
\begin{enumerate}
\item \label{itm:changes1} Ekstra mulighed for godkendelse af vedtægtsændringer. (Bilag~\ref{app:b1})
\item \label{itm:changes2} Ændring af titlen »ledelsen« til »bestyrelsen«. (Bilag~\ref{app:b2})
\item \label{itm:changes3} Valg af bestyrelsen/revybosser. (Bilag~\ref{app:b3})
\item \label{itm:changes4} Klargøring af hvad en stemmeberettiget person er. (Bilag~\ref{app:b4})
\item \label{itm:changes5} Gøre det muligt for at stemme in absentia med fuldmagt. (Bilag~\ref{app:b5})
\end{enumerate}
\item Eventuelt.
\end{enumerate}

Forslag til generalforsamlingen sendes til boss@dikurevy.dk senest 7
dage før generalforsamlingen til ledelsens godkendelse jf. § 5.
Ledelsen fremsætter herefter en endelig dagsorden, som
offentliggøres senest 2 dage før generalforsamlingen.

\newpage

\section*{Uddybende}

\subsection*{\ref{itm:changes}. Vedtægtsændringer}

Ledelsen vil gerne indstille følgende vedtægtsændringer til foreningens
godkendelse:

\subsubsection*{\ref{itm:changes1} Ekstra mulighed for godkendelse af vedtægtsændringer}

Som det er nu skal vedtægtsændringer foregå på en ordinær
generalforsamling. Det kræver 2/3 af stemmerne med minimum 15 stemmer
for vedtægtsændringerne. Selv hvis 2/3 af de fremmødte stemmer for, 
men der ikke er minimum 15 stemmer, så blive vedtægtsændringerne
ikke vedtaget.

Ledelsen forslår derfor en tilføjelse til § 10, der gør det muligt at
kalde til en ekstraordinær generalforsamling efter den ordinære
generalforsamling for at stemme med et kvalificeret flertal for de
vedtægtsændringer der fik 2/3 af de fremmødtes stemmer til den ordinære
generalforsamling, men ikke opfulgte minimumskravet.  Det kræver dog at
minimum 10 medlemmer ved tilstede ved den ordinære generalforsamling.

(Se bilag~\ref{app:b1} for de konkrete vedtægtsændringer.)

\subsubsection*{\ref{itm:changes2} Ændring af titlen »ledelsen« til »bestyrelsen«}

I de nuværende vedtægter omtales det der reelt er en bestyrelse, som
»ledelsen«.  Ledelsen i foreningsmæssig sammenhæng har ingen reel
betydning, idet at en bestyrelse er ledelsen \emph{mellem}
generalforsamlinger, idet at på en generalforsamling er \emph{alle}
medlemmer af foreningen ledelsen.

Der er intet krav for hvad en bestyrelse skal indeholde (hverken en
formand eller en kasserer), der skal blot \emph{være} en bestyrelse.

(Se bilag~\ref{app:b2} for de konkrete vedtægtsændringer.)

\subsubsection*{\ref{itm:changes3} Valg af bestyrelsen/revybosser}

Ved krisemødet 9/3, valgte et overvældende flertal (13 mod 2), at
revybosserne nominerer en ny bestyrelse, som skal godkendes af
revygruppen/foreningen.

På krisemødet var afstemningerne blot vejledende, altså ikke
bindende, og det bliver de ikke før de kommer i vedtægterne.

(Se bilag~\ref{app:b3} for de konkrete vedtægtsændringer.)

\subsubsection*{\ref{itm:changes4} Klargøring af hvad en stemmeberettiget
person er}

I de nuværende vedtægter under § 7 er en person stemmeberettiget hvis
vedkommende har været medlem af foreningen inden for det sidste år og er
tilstede på generalforsamlingen.

Problemet er at teksten tillader en hver person som har været medlem på
et givent tidspunkt inden for det sidste år at stemme, også folk der er
meldt/smidt ud på det tidspunkt generalforsamlingen finder sted.

Derfor forslås en ændring af § 7, således at man skal være medlem af
foreningen på det tidspunkt generalforsamlingen finder sted, og
selvfølgelig være tilstede.

(Se bilag~\ref{app:b4} for de konkrete vedtægtsændringer.)

\subsubsection*{\ref{itm:changes5} Gøre det muligt for at stemme in absentia
med fuldmagt}

Kan man ikke deltage til generalforsamlingen, som det er nu, er det
ikke muligt at stemme.  Derfor dette forslag, som gør det muligt at
stemme in absentia, såfremt at man har givet en anden person lov
til det.

En fuldmagt sendes til \url{boss@dikurevy.dk} umiddelbart før
generalforsamlingen finder sted.

(Se bilag~\ref{app:b5} for de konkrete vedtægtsændringer.)

\newpage

\begin{appendices}
\chapter{Ekstra mulighed for godkendelse af vedtægtsændringer}
\label{app:b1}

\section*{Teksttilføjelse til § 10:}

\begin{quotationb}
Stk. 1 Vedtagelse på ekstraordinær generalforsamling med kvalificeret flertal\\
\\
Hvis en ændring til vedtægterne vedtages på den ordinære generalforsamling,
hvor minimum 10 stemmeberettigede var tilstedeværende, med 2/3 af de
stemmeberettigedes accept, men uden at opfylde kravet om minimum 15 stemmer,
kan bestyrelsen indkalde til en ekstraordinær generalforsamling om samme
vedtægtsændringer, hvorved de kan vedtages med 2/3 af de fremmødte
stemmeberettigedes accept uden minimumskrav.
\end{quotationb}

\chapter{Ændring af titlen »ledelsen« til »bestyrelsen«}
\label{app:b2}

\section*{Tekstændringer til § 4:}

\paragraph{Afsnit 2}

\begin{quotationb}
Ekskludering eller udmelding af forening foregår ved afmeldelse af mailinglisten
revy@dikurevy.dk.  Mailinglisten administreres til enhver tid af bestyrelsen.
\end{quotationb}

\section*{Tekstændringer til § 5:}

\paragraph{Punkt 1}

\begin{quotationb}
Bestyrelsens beretning om foreningens gang det forløbne år.
\end{quotationb}

\paragraph{Punkt 2}

\begin{quotationb}
Bestyrelsen fremlægger det nye års arbejdsmetoder, som er baseret på årets
evalueringer.
\end{quotationb}

\paragraph{Afsnit 3}

\begin{quotationb}
Forslag til generalforsamlingen sendes til boss@dikurevy.dk senest 7 dage før
generalforsamlingen til bestyrelsens godkendelse.  Bestyrelsen fremsætter
herefter en endelig dagsorden, som offentliggøres senest 2 dage før
generalforsamlingen.
\end{quotationb}

\section*{Tekstændringer til § 6:}

\paragraph{Afsnit 1}

\begin{quotationb}
Har bestyrelsen behov for at annoncere ekstraordinære meddelelser indkaldes
der til ekstraordinær generalforsamling med mindst 14 dages varsel.
\end{quotationb}

\section*{Tekstændringer til § 8:}

\paragraph{Titel}

\begin{quotationb}
Bestyrelsen
\end{quotationb}

\paragraph{Afsnit 1}

\begin{quotationb}
Bestyrelsen består af revybosserne.  Revybosserne består af 2-3 medlemmer,
som har en ekstra interesse i at føre revyen videre.  Bestyrelsen er
selvsupplerende.
\end{quotationb}

\paragraph{Afsnit 2}

\begin{quotationb}
Bestyrelsen har til ansvar at tage beslutninger i foreningen bedste, og ikke
i personlige interesse, når der opstår opgaver, stillingsspørgsmål og andet
vedrørende foreningen.  Ligeledes har de pligt til at lytte til foreningens
øvrige medlemmer.
\end{quotationb}

\section*{Tekstændringer til § 9 stk. 2:}

\paragraph{Afsnit 1}

\begin{quotationb}
Varetagelsen af foreningens konto foretages af bestyrelsen, nærmere betegnet
den økonomiansvarlige i bestyrelsen.  Den økonomiansvarlige har til pligt at
sørge for, at foreningens formue forvaltes på bedste vis.
\end{quotationb}

\section*{Tekstændringer til § 9 stk. 3:}

\paragraph{Afsnit 1}

\begin{quotationb}
Foreningen hæfter kun for sine forpligtelser med den af foreningen til enhver
tid tilhørende formue, herunder foreningens ejendomme.  Der påhviler ikke
foreningens medlemmer eller bestyrelsen nogen personlig hæftelse.
\end{quotationb}

\section*{Tekstændringer til Bilag A:}

\paragraph{Afsnit 2}

\begin{quotationb}
Bestyrelsen kan ændre de nuværende mailadresser uden om en generalforsamling,
såfremt alle nuværende medlemmer på de nuværende mailadresser, flyttes med
over på de nye, når disse er på plads.
\end{quotationb}

\paragraph{Afsnit 3}

\begin{quotationb}
Bestyrelsen kan ændre navnet på dokumentet i fodnote 1, når dokumentet er
færdigudarbejdet.
\end{quotationb}

\chapter{Valg af bestyrelsen/revybosser}
\label{app:b3}

\section*{Tekstændringer til § 8}

\paragraph{Afsnit 1}

\begin{quotationb}
Bestyrelsen består af revybosserne.  Revybosserne består af 2-3 medlemmer,
som har en ekstra interesse i at føre revyen videre.
\end{quotationb}

\setlength{\parskip}{15pt}
(Teksten »Bestyrelsen er selvsupplerende.« er blevet fjernet.)
\setlength{\parskip}{0pt}

\section*{Teksttilføjelse til § 8}

\begin{quotationb}
Stk. 1 Nominering og godkendelse af ny bestyrelse\\
\\
På den ordinære generalforsamling, nominerer den nuværende bestyrelse en ny
bestyrelse.  Denne bestyrelse skal godkendes ved simpelt flertal.  En ny
bestyrelse kan også nomineres på en ekstraordinær generalforsamling.\\
\\
Hvis den nye bestyrelse ikke godkendes, kan den nuværende bestyrelse foretage
nye bestyrelsesnomineringer to ekstra gange.  Fejler også disse nomineringer,
overgår valget af bestyrelse til demokratisk valg, se stk. 2.\\
\\
Stk. 2 Demokratisk valg af ny bestyrelse\\
\\
Hvis bestyrelsens nomineringer af en ny bestyrelse ikke bliver godkendt efter
tre forsøg, overgår valget af bestyrelsen til demokratisk valg.\\
\\
Ethver medlem af foreningen kan opstille sig selv til bestyrelsen.
Medlemmer bliver godkendt ved simpelt flertal.
\end{quotationb}

\section*{Teksttilføjelse til § 5:}

\paragraph{Mellem Punkt 1 og 2}

\begin{quotationb}
2. Bestyrelsen nominerer en ny bestyrelse, jf. § 8 stk. 1.
\end{quotationb}

\setlength{\parskip}{15pt}
(De efterfølgende punkter får deres værdi øget med 1.)
\setlength{\parskip}{0pt}

\chapter{Klargøring af hvad en stemmeberettiget person er}
\label{app:b4}

\section*{Tekstændringer til § 7}

\paragraph{Afsnit 1}

\begin{quotationb}
Til generalforsamlingen træffes beslutninger ved simpelt flertal, hvilket
vil sige mere end halvdelen af de afgivne stemmer. Stemmeberettigede
er personer, der på dagen for generalforsamlingen er medlem af 
foreningen, og som er tilstede ved generalforsamlingen. Dette 
dokumenteres ved hjælp af de udtræk, der foretages hver måned, 
jf. § 4. Beslutninger kan foretages ved håndsoprækning. Hvis
nogen ønsker det, skal der foretages skriftlig afstemning.
\end{quotationb}

\setlength{\parskip}{15pt}
(Yderligere er »punkt« blevet til »§« og »handsoprækning« til
»håndsoprækning«.)
\setlength{\parskip}{0pt}

\chapter{Gøre det muligt for at stemme in absentia med fuldmagt}
\label{app:b5}

\section*{Tekstændringer til § 7}

\paragraph{Afsnit 1}

\begin{quotationb}
Til generalforsamlingen træffes beslutninger ved simpelt flertal, hvilket
vil sige mere end halvdelen af de afgivne stemmer. Stemmeberettigede
er personer, der på dagen for generalforsamlingen er medlem af 
foreningen, og som er tilstede ved generalforsamlingen, enten fysisk eller
via fuldmagt, jf. stk. 2. Dette dokumenteres ved hjælp af de udtræk, der
foretages hver måned, jf. § 4. Beslutninger kan foretages ved håndsoprækning.
Hvis nogen ønsker det, skal der foretages skriftlig afstemning.
\end{quotationb}

\section*{Teksttilføjelse til § 7}
\begin{quotationb}
Stk. 2 Mulighed for at stemme in absentia med fuldmagt\\
\\
Kan et medlem ikke deltage til en generalforsamling, er det muligt for medlemmet,
at sende en anden person til at stemme for medlemmet.  Dette kræver en
fuldmagt.  En fuldmagt skal sendes til \url{boss@dikurevy.dk}.  Fuldmagten skal
være i bestyrelsens hænde umiddelbart før generalforsamlingen finder sted.\\
\\
Fuldmagten skal indeholde medlemmets navn, navn på personen der stemmer på
medlemmets vegne, og hvilken generalforsamlingen der er tale om.  Emailadressen,
fuldmagten er sendt fra, skal være den samme som medlemmets emailadresse på
mailinglisten, jf. § 4.\\
\\
En person der stemmer på andres vegne behøver ikke at være medlem af foreningen.
En person må maksimalt stemme på 3 medlemmers vegne udover sit eget.  Hvis
personen ikke er medlem, er begræsningen stadig 3 medlemmer.\\
\\
Et stemme in absentia har samme vægt som et stemme fra en tilstedeværende.
\end{quotationb}

\end{appendices}

\end{document}

