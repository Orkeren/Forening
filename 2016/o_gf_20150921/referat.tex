\documentclass[a4paper,11pt]{report}

\usepackage[utf8]{inputenc}
\usepackage[T1]{fontenc}
\usepackage[danish]{babel}
\usepackage{url}
\usepackage[toc,page]{appendix}
\renewcommand{\appendixtocname}{Bilag}
\renewcommand{\appendixpagename}{Bilag}

\newenvironment{quotationb}%
{\begin{tabular}{|p{10cm}}}%
{\\\end{tabular}}

\begin{document}

\section*{Referat for ordinær generalforsamling d. 21. september 2015}

\subsection*{Til stede}

\begin{itemize}
\item Kasper Elbo
\item Jenny-Margrethe Vej
\item Troels Henriksen
\item Lea Brun
\item Claus Martin Jürgensen Appel
\item Simon Skjerning Linneberg
\item Niels Gustav Westphal Serup
\item Tor Skovsgaard
\item Amanda Elizabeth Riberholt-Rischel
\item Maya Saietz
\item Nanna Elisabeth Vagner Bernbom
\item Arinbjörn Brandsson
\item Sven Uhrenholdt Frenzel
\item Nana Girotti
\item Mikkel Kjær Jensen
\item Mikkel Storgaard Knudsen
\item Søren Pilgård
\item Caroline Miller
\item David Rechnagel Udsen
\item Oskar Behrendt
\item René Løwe
\end{itemize}

\newpage

\section*{Valg af dirigent}

David Rechnagel Udsen er dirigent.  Niels Gustav Westphal Serup er referent.

\section*{Fremlæggelse af foreningens gang det forløbne år}

Siden sidste generalforsamling: Det har været et dårligt år.

De nuværende bosser -- David Rechnagel Udsen, Nana Girotti og Mikkel Storgaard
Knudsen -- og Jenny-Margrethe Vej taler.

Der skete ikke meget i det første halve års tid.  Der var bl.a. en hyttetur.  De
gamle bosser trådte af efter krisemøde.  David Rechnagel udsen og Nana Girotti
blev bosser efter Daniel Egeberg og Maya Saietz trådte af.

Målet efter krisemødet var at få årets revy til at spille.  Der var ikke tid til
at finde ud af hvad, om noget, man skulle gøre anderledes.

Det blev en revy!

Efter revyen er der endnu ikke sket så meget.  Nu begynder et nyt år!

\section*{Opfølgning fra krisemødet og postrevyen}

Se Ejnars referat.

David og Nana blev de nye bosser.

Der blev bl.a. vedtaget en ny boss-vedtagnings-procedure: Ved generalforsamlingen
foreslår de siddende bosser de nye bosser, og generalforsamlingen godkender ved
simpelt flertal.

\section*{Regnskab}

Fra bestyrelsens side ønsker man at udskyde dette punkt til en ekstraordinær
generalforsamling inden for seks måneder.  Årsagen er især at det har været en
udfordring at få overdraget ansvar for revyens regnskab fra den tidligere
bestyrelse.

\section*{Budget for revyåret 2016}

Dette udskydes også til en ekstraordinær generalforsamling.

\newpage

\section*{Vedtægtsændringer}

\subsection*{Ekstra mulighed for godkendelse af vedtægtsændringer}

Søren Pilgård mener at det er uklart at der står "vedtages" i forbindelse med
generalforsamlingen, da det ikke handler om vedtagelse, men om mulighed for
vedtagelse på den ekstraordinære generalforsamling.

Amanda mener at det er klart nok.

Der bliver foreslået at fjerne ordet "vedtages".  Sven har et forslag til en
omformulering med samme betydning.

\subsubsection*{Stemmer}
\begin{tabular}{ c | c | c }
For & Imod & Ikke afgivende \\
\hline
20 & 0 & 1 \\
\end{tabular}

Forslaget er vedtaget, med den note at formuleringen skal forbedres jf. Svens
forslag.

Se bilag \ref{app:change1} for de endelige ændringer.

\subsection*{Ændring af titlen »ledelsen« til »bestyrelsen«}

David har kontaktet et bureau for foreningsvejledning og har fået at vide at
DIKUrevyen skal have en bestyrelse i kraft af sin tilstand.  Det er nemmere at
kommunikere med offentlige instanser når vi har noget der hedder en
"bestyrelse".

Alle i bestyrelsen har formandsmagt, og en i bestyrelsen er økonomiansvarlig.

Søren foreslår at der tilføjes en sætning "Bestyrelsen omtales i det daglige som
revybosser."  Troels foreslår helt at droppe at der bliver nævnt "revybosser" i
vedtægterne og omtaler det konsistent som "bestyrelsen".  Sven har et forslag
til en omformulering der bruger de to ændringsforslag.

\subsubsection*{Stemmer}
\begin{tabular}{ c | c | c }
For & Imod & Ikke afgivende \\
\hline
21 & 0 & 0 \\
\end{tabular}

Forslaget er vedtaget, med den note at formuleringen skal forbedres jf. Svens
forslag.

Se bilag \ref{app:change2} for de endelige ændringer.

\subsection*{Valg af bestyrelsen/revybosser}

Der pointeres at nominering skal foregå på den samme generalforsamling, for der
skal vælges en ny bestyrelse.

Der påpeges at hvis bestyrelsen til generalforsamlingen ikke er i stand til at
påpege en ny bestyrelse, overgår bestyrelsesvalg til valg ved simpelt flertal.

Der bliver talt om hvad der sker hvis en bestyrelse går af og ingen stiller op.

Opsummering: Der er stemning for at gøre vedtægterne mindre vage på dette
område, men sådanne ændringer virker for store til at vi kan tilføje dem i denne
afstemning.  I fremtiden skal der være valg af vedtægtsændringer før valg af
bestyrelsen.

Der foreslås at vi stemmer om vedtagelse af ændringen som den står, da den trods
alt er bedre end i de nuværende vedtægter.  Vedtægterne i dette område skal så
gøres mere klare til næste generalforsamling.

\subsubsection*{Stemmer}
\begin{tabular}{ c | c | c }
For & Imod & Ikke afgivende \\
\hline
20 & 1 & 0 \\
\end{tabular}

Forslaget er vedtaget.

Se bilag \ref{app:change3} for de endelige ændringer.

\subsection*{Klargøring af hvad en stemmeberettiget person er}

Der påpeges at bestyrelsen med den nuværende formulering kan ekskludere
medlemmer op til en generalforsamling og dermed stoppe deres stemmemuligheder.

Der foreslås at forslaget trækkes tilbage, da der mangler detaljer om specialtilfælde,
og at der laves et mere udpenslet ændringsforslag til næste generalforsamling.

Bestyrelsen trækker forslaget tilbage.

\subsection*{Gøre det muligt for at stemme in absentia med fuldmagt}

Der foreslås at der bare kræves almindelig fuldmagt med papir fremfor at sende
den til boss@dikurevy.dk.  Det gør det også kortere.  De nuværende bosser kommer
med en bedre formulering.

\subsubsection*{Stemmer}
\begin{tabular}{ c | c | c }
For & Imod & Ikke afgivende \\
\hline
20 & 0 & 1 \\
\end{tabular}

Forslaget er vedtaget, med den note at formuleringen skal forbedres
jf. bossernes forslag.

Se bilag \ref{app:change5} for de endelige ændringer.

\section*{Eventuelt}

\subsection*{Eventuelt A: Vedtægtsændringsforslagstilføjelse: Mulighed for overgangsbestyrelse}

Der påpeges at en mulig overgangsbestyrelse skal have mulighed for at sige ja
eller nej til at være overgangsbestyrelse.

Der er for meget uenighed om detaljerne ved ændringsforslaget, så bestyrelsen
trækker forslaget tilbage.

\subsection*{Eventuelt B: Vedtægtsændringsforslagstilføjelse: Undgå misbrug af ekskludering før generalforsamling}

Der foreslås at dette forslag udskydes til næste generalforsamling for at kunne
finpudse det.  Bestyrelsen trækker forslaget tilbage.

\section*{Nominering af ny bestyrelse}

Den siddende bestyrelse nominerer sig selv:

\begin{itemize}
\item David Rechnagel Udsen
\item Nana Girotti
\item Mikkel Storgaard Knudsen
\end{itemize}

Der afholdes hemmelig afstemning for en god ordens skyld.  Jenny-Margrethe Vej
er stemmetæller.

\subsubsection*{Stemmer}
\begin{tabular}{ c | c | c }
For & Imod & Ikke afgivende \\
\hline
20 & 0 & 1 \\
\end{tabular}

Bestyrelsen er vedtaget!

\begin{appendices}

\chapter{Ekstra mulighed for godkendelse af vedtægtsændringer}
\label{app:change1}

\section*{Teksttilføjelse til § 10:}

\begin{quotationb}
Stk. 1 Vedtagelse på ekstraordinær generalforsamling med kvalificeret flertal\\
\\
Hvis der er 2/3 flertal for en ændring af vedtægterne på den ordinære
generalforsamling, hvor minimum 10 stemmeberettigede er tilstedeværende, men
uden at opfylde kravet om minimum 15 stemmer, kan bestyrelsen indkalde til en
ekstraordinær generalforsamling om samme vedtægtsændring, hvorved den kan 
vedtages med 2/3 af de fremmødte stemmeberettigedes accept uden minimumskrav.
\end{quotationb}

\chapter{Ændring af titlen »ledelsen« til »bestyrelsen«}
\label{app:change2}

\section*{Tekstændringer til § 4:}

\paragraph{Afsnit 2}

\begin{quotationb}
Ekskludering eller udmelding af forening foregår ved afmeldelse af mailinglisten
revy@dikurevy.dk.  Mailinglisten administreres til enhver tid af bestyrelsen.
\end{quotationb}

\section*{Tekstændringer til § 5:}

\paragraph{Punkt 1}

\begin{quotationb}
Bestyrelsens beretning om foreningens gang det forløbne år.
\end{quotationb}

\paragraph{Punkt 2}

\begin{quotationb}
Bestyrelsen fremlægger det nye års arbejdsmetoder, som er baseret på årets
evalueringer.
\end{quotationb}

\paragraph{Afsnit 3}

\begin{quotationb}
Forslag til generalforsamlingen sendes til \url{boss@dikurevy.dk} senest 7 dage før
generalforsamlingen til bestyrelsens godkendelse.  Bestyrelsen fremsætter
herefter en endelig dagsorden, som offentliggøres senest 2 dage før
generalforsamlingen.
\end{quotationb}

\section*{Tekstændringer til § 6:}

\paragraph{Afsnit 1}

\begin{quotationb}
Har bestyrelsen behov for at annoncere ekstraordinære meddelelser indkaldes
der til ekstraordinær generalforsamling med mindst 14 dages varsel.
\end{quotationb}

\section*{Tekstændringer til § 8:}

\paragraph{Titel}

\begin{quotationb}
Bestyrelsen
\end{quotationb}

\paragraph{Afsnit 1}

\begin{quotationb}
Bestyrelsen består af 2-3 medlemmer, som har en ekstra interesse i at føre
revyen videre.  Bestyrelsen er selvsupplerende.
\end{quotationb}

\paragraph{Afsnit 2}

\begin{quotationb}
Bestyrelsen har til ansvar at tage beslutninger i foreningen bedste, og ikke
i personlige interesse, når der opstår opgaver, stillingsspørgsmål og andet
vedrørende foreningen.  Ligeledes har de pligt til at lytte til foreningens
øvrige medlemmer.
\end{quotationb}

\section*{Teksttilføjelse til § 8:}

\begin{quotationb}
Bestyrelsen omtales til dagligt som Revybosserne.
\end{quotationb}

\section*{Tekstændringer til § 9 stk. 2:}

\paragraph{Afsnit 1}

\begin{quotationb}
Varetagelsen af foreningens konto foretages af bestyrelsen, nærmere betegnet
den økonomiansvarlige i bestyrelsen.  Den økonomiansvarlige har til pligt at
sørge for, at foreningens formue forvaltes på bedste vis.
\end{quotationb}

\section*{Tekstændringer til § 9 stk. 3:}

\paragraph{Afsnit 1}

\begin{quotationb}
Foreningen hæfter kun for sine forpligtelser med den af foreningen til enhver
tid tilhørende formue, herunder foreningens ejendomme.  Der påhviler ikke
foreningens medlemmer eller bestyrelsen nogen personlig hæftelse.
\end{quotationb}

\section*{Tekstændringer til Bilag A:}

\paragraph{Afsnit 2}

\begin{quotationb}
Bestyrelsen kan ændre de nuværende mailadresser uden om en generalforsamling,
såfremt alle nuværende medlemmer på de nuværende mailadresser, flyttes med
over på de nye, når disse er på plads.
\end{quotationb}

\paragraph{Afsnit 3}

\begin{quotationb}
Bestyrelsen kan ændre navnet på dokumentet i fodnote 1, når dokumentet er
færdigudarbejdet.
\end{quotationb}

\chapter{Valg af bestyrelsen/revybosser}
\label{app:change3}

\section*{Tekstændringer til § 8}

\paragraph{Afsnit 1}

\begin{quotationb}
Bestyrelsen består af 2-3 medlemmer, som har en ekstra interesse i at føre
revyen videre.
\end{quotationb}

\section*{Teksttilføjelse til § 8}

\begin{quotationb}
Stk. 1 Nominering og godkendelse af ny bestyrelse\\
\\
På den ordinære generalforsamling, nominerer den nuværende bestyrelse en ny
bestyrelse.  Denne bestyrelse skal godkendes ved simpelt flertal.  En ny
bestyrelse kan også nomineres på en ekstraordinær generalforsamling.\\
\\
Hvis den nye bestyrelse ikke godkendes, kan den nuværende bestyrelse
foretage nye bestyrelsesnomineringer to ekstra gange.  Alle nomineringsforsøgene
skal foregå på den samme generalforsamling, da der skal godkendes en ny
bestyrelse.  Fejler også disse nomineringer, overgår valget af bestyrelse til
demokratisk valg, se stk. 2.\\
\\
Stk. 2 Demokratisk valg af ny bestyrelse\\
\\
Hvis bestyrelsens nomineringer af en ny bestyrelse ikke bliver godkendt efter
tre forsøg, overgår valget af bestyrelsen til demokratisk valg.\\
\\
Ethvert medlem af foreningen kan opstille sig selv til bestyrelsen.
Medlemmer bliver godkendt ved simpelt flertal.
\end{quotationb}

\section*{Teksttilføjelse til § 5:}

\paragraph{Mellem Punkt 4 og 5}

\begin{quotationb}
5. Bestyrelsen nominerer en ny bestyrelse, jf. § 8 stk. 1.
\end{quotationb}

\setlength{\parskip}{15pt}
(Det efterfølgende punkt får dets værdi øget med 1.)
\setlength{\parskip}{0pt}

\chapter{Gøre det muligt for at stemme in absentia med fuldmagt}
\label{app:change5}

\section*{Tekstændringer til § 7}

\paragraph{Afsnit 1}

\begin{quotationb}
Til generalforsamlingen træffes beslutninger ved simpelt flertal, hvilket
vil sige mere end halvdelen af de afgivne stemmer. Stemmeberettigede er
personer der, inden for det sidste år, har været medlem af foreningen, og som er
til stede ved generalforsamlingen, enten fysisk eller via fuldmagt, jf. stk. 2.
Beslutninger kan foretages ved håndsoprækning. Hvis nogen ønsker det, skal der
foretages skriftlig afstemning.
\end{quotationb}

\section*{Teksttilføjelse til § 7}
\begin{quotationb}
Stk. 2 Mulighed for at stemme in absentia med fuldmagt\\
\\
Kan et medlem ikke deltage til en generalforsamling, er det muligt for 
medlemmet, at sende en anden person til at stemme for medlemmet. Dette kræver en
fuldmagt. Fuldmagten afleveres til bestyrelsen ved begyndelsen af
generalforsamlingen. Fuldmagten skal være skrevet og underskrevet i blæk og være
skrevet på et A4-papir.\\
\\
Fuldmagten skal indeholde medlemmets navn, navn på personen, der stemmer på
medlemmets vegne, og hvilken generalforsamling, der er tale om. Mailadressen, 
som medlemmet er registreret på \url{revy@dikurevy.dk} med, skal også fremgå af
fuldmagten. Den fuldmægtige skal dokumentere sin identitet ved hjælp af gyldig
billedlegitimation.\\
\\
En person der stemmer på andres vegne behøver ikke at være medlem af foreningen.
En person må maksimalt stemme på 3 medlemmers vegne udover sit eget.  Hvis
personen ikke er medlem, er begrænsningen stadig 3 medlemmer.\\
\\
Et stemme in absentia har samme vægt som et stemme fra en tilstedeværende.
\end{quotationb}

\end{appendices}

\end{document}
